\begin{abstract}\\
Through this manual, we learn how to communicate between SPI, Wishbone Interfacing and Address Mapping.
On the Vaman Board, we have an EOS S3 and ESP32. The Communication between these two happens via SPI i.e, Serial Peripheral Interface.And this is facilitated only when all the 4 jumpers on the board are closed.
\end{abstract}


\section{Components Table}
\begin{enumerate}[label=\thesection.\arabic*.,ref=\thesection.\theenumi]
\numberwithin{equation}{enumi}
\numberwithin{figure}{enumi}
\numberwithin{table}{enumi}
\item In addition to the components referred in \tabref{Tab:components} we require a Seven Segment Display module.
\subsection{Vaman Board ESP32}
\item Refer the Vaman Board  ESP32 \figref{fig:vaman-pin_sheet}
\subsection{USB UART}
\item Refer the USB-UART \figref{fig:usb-uart}

\section{Circuit Connections}
\item  Make the Circuit Connections as per the \tabref{Tab:ble-connections}. 
\item Connect the USB-UART pins to the Vaman ESP32 pins according to \tabref{Tab:connections4}.
\item Download the “dabble” application from the play store on an Android phone.
\item Using dabble application, connect to the ESP32 on the seven segment using Bluetooth connection.
\item Control the seven segment using the GUI controls on the dabble application. 

\section{Code Execution For Wifi Toycar}
\raggedright
\item Now execute the following code
\begin{lstlisting}
vaman/seven-seg/codes/wifi/src/main.cpp
\end{lstlisting}

\item Build the ESP32 firmware
\begin{lstlisting}
cd vaman/seven-seg/codes/wifi
pio run
\end{lstlisting} 

\item Flash ESP32 firmware ( connect USB-UART adapter )
\begin{lstlisting}
pio run -t upload
\end{lstlisting} 

\item Connect your own TAB /Phone Hot spot and  Enter Your SSID and  Password
\begin{lstlisting}
const char* ssid = "fwc";         /*Enter Your SSID*/ 
const char* password = "fwc123"; /*Enter Your Password*
\end{lstlisting} 
\item Install the \textbf{Dabble app} on the Mobile from the \textbf{Playstore}. Connect it to the \textbf{ESP32} on the Vaman Board using \textbf{WiFi}. Change the controls to \textbf{Joystick mode} to navigate the UGV.\\

%\section{Code Execution for Bluetooth Toycar}
%\raggedright
%\item Now,Execute the following code 
%
%\begin{lstlisting}
%vaman/toycar/codes/bluetooth_toycar
%\end{lstlisting}
%
%\item Build the ESP32 firmware
%\begin{lstlisting}
%cd esp32_pwmctrl
%pio run
%\end{lstlisting} 
%
%\item Flash ESP32 firmware ( connect USB-UART adapter )
%\begin{lstlisting}
%pio run -t nobuild -t upload
%\end{lstlisting} 
%
%\item If using termux, send .pio/build/esp32doit-devkit-v1/firmware.bin to PC using
%\begin{lstlisting}
%scp .pio/build/esp32doit-devkit-v1/firmware.bin Username@IPAddress:
%\end{lstlisting} 
%
%\item  Modify line 140 of config.mk to setup path to pygmy-sdk and then Build m4 firmware using
%\begin{lstlisting}
%cd m4_pwmctrl/GCC_Project
%make
%\end{lstlisting}
%
%\item If using termux, send output/m4{\_}pwmctrl.bin to PC using
%\begin{lstlisting}
%scp output/m4_pwmctrl.bin username@IPaddress:
%\end{lstlisting} 
%
%\item Build fpga source (.bin file)
%\begin{lstlisting}
%cd fpga_pwmctrl/rtl
%ql_symbiflow -compile -d ql-eos-s3 -P pu64 -v *.v -t AL4S3B_FPGA_Top -p quickfeather.pcf -dump jlink binary 
%\end{lstlisting} 
%
%\item If using termux, send AL4S3B{\_}FPGA{\_}Top.bin to PC using
%\begin{lstlisting}
%scp AL4S3B_FPGA_Top.bin username@IPaddress:
%\end{lstlisting} 
%
%\item Connect usb cable to vaman board and Flash eos s3 soc, using
%\begin{lstlisting}
%sudo python3 <Type path to tiny fpga programmer application> --port /dev/ttyACM0  --appfpga AL4S3B_FPGA_Top.bin --m4app m4_pwmctrl.bin --mode m4-fpga --reset
%\end{lstlisting} 
%
%\item Install the \textbf{Dabble app} on the Mobile from the \textbf{Playstore}. Connect it to the \textbf{ESP32} on the Vaman Board using \textbf{Bluetooth}. Change the controls to \textbf{Joystick mode} to navigate the UGV.\\


\end{enumerate}
